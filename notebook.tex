
% Default to the notebook output style

    


% Inherit from the specified cell style.




    
\documentclass[11pt]{article}

    
    
    \usepackage[T1]{fontenc}
    % Nicer default font (+ math font) than Computer Modern for most use cases
    \usepackage{mathpazo}

    % Basic figure setup, for now with no caption control since it's done
    % automatically by Pandoc (which extracts ![](path) syntax from Markdown).
    \usepackage{graphicx}
    % We will generate all images so they have a width \maxwidth. This means
    % that they will get their normal width if they fit onto the page, but
    % are scaled down if they would overflow the margins.
    \makeatletter
    \def\maxwidth{\ifdim\Gin@nat@width>\linewidth\linewidth
    \else\Gin@nat@width\fi}
    \makeatother
    \let\Oldincludegraphics\includegraphics
    % Set max figure width to be 80% of text width, for now hardcoded.
    \renewcommand{\includegraphics}[1]{\Oldincludegraphics[width=.8\maxwidth]{#1}}
    % Ensure that by default, figures have no caption (until we provide a
    % proper Figure object with a Caption API and a way to capture that
    % in the conversion process - todo).
    \usepackage{caption}
    \DeclareCaptionLabelFormat{nolabel}{}
    \captionsetup{labelformat=nolabel}

    \usepackage{adjustbox} % Used to constrain images to a maximum size 
    \usepackage{xcolor} % Allow colors to be defined
    \usepackage{enumerate} % Needed for markdown enumerations to work
    \usepackage{geometry} % Used to adjust the document margins
    \usepackage{amsmath} % Equations
    \usepackage{amssymb} % Equations
    \usepackage{textcomp} % defines textquotesingle
    % Hack from http://tex.stackexchange.com/a/47451/13684:
    \AtBeginDocument{%
        \def\PYZsq{\textquotesingle}% Upright quotes in Pygmentized code
    }
    \usepackage{upquote} % Upright quotes for verbatim code
    \usepackage{eurosym} % defines \euro
    \usepackage[mathletters]{ucs} % Extended unicode (utf-8) support
    \usepackage[utf8x]{inputenc} % Allow utf-8 characters in the tex document
    \usepackage{fancyvrb} % verbatim replacement that allows latex
    \usepackage{grffile} % extends the file name processing of package graphics 
                         % to support a larger range 
    % The hyperref package gives us a pdf with properly built
    % internal navigation ('pdf bookmarks' for the table of contents,
    % internal cross-reference links, web links for URLs, etc.)
    \usepackage{hyperref}
    \usepackage{longtable} % longtable support required by pandoc >1.10
    \usepackage{booktabs}  % table support for pandoc > 1.12.2
    \usepackage[inline]{enumitem} % IRkernel/repr support (it uses the enumerate* environment)
    \usepackage[normalem]{ulem} % ulem is needed to support strikethroughs (\sout)
                                % normalem makes italics be italics, not underlines
    

    
    
    % Colors for the hyperref package
    \definecolor{urlcolor}{rgb}{0,.145,.698}
    \definecolor{linkcolor}{rgb}{.71,0.21,0.01}
    \definecolor{citecolor}{rgb}{.12,.54,.11}

    % ANSI colors
    \definecolor{ansi-black}{HTML}{3E424D}
    \definecolor{ansi-black-intense}{HTML}{282C36}
    \definecolor{ansi-red}{HTML}{E75C58}
    \definecolor{ansi-red-intense}{HTML}{B22B31}
    \definecolor{ansi-green}{HTML}{00A250}
    \definecolor{ansi-green-intense}{HTML}{007427}
    \definecolor{ansi-yellow}{HTML}{DDB62B}
    \definecolor{ansi-yellow-intense}{HTML}{B27D12}
    \definecolor{ansi-blue}{HTML}{208FFB}
    \definecolor{ansi-blue-intense}{HTML}{0065CA}
    \definecolor{ansi-magenta}{HTML}{D160C4}
    \definecolor{ansi-magenta-intense}{HTML}{A03196}
    \definecolor{ansi-cyan}{HTML}{60C6C8}
    \definecolor{ansi-cyan-intense}{HTML}{258F8F}
    \definecolor{ansi-white}{HTML}{C5C1B4}
    \definecolor{ansi-white-intense}{HTML}{A1A6B2}

    % commands and environments needed by pandoc snippets
    % extracted from the output of `pandoc -s`
    \providecommand{\tightlist}{%
      \setlength{\itemsep}{0pt}\setlength{\parskip}{0pt}}
    \DefineVerbatimEnvironment{Highlighting}{Verbatim}{commandchars=\\\{\}}
    % Add ',fontsize=\small' for more characters per line
    \newenvironment{Shaded}{}{}
    \newcommand{\KeywordTok}[1]{\textcolor[rgb]{0.00,0.44,0.13}{\textbf{{#1}}}}
    \newcommand{\DataTypeTok}[1]{\textcolor[rgb]{0.56,0.13,0.00}{{#1}}}
    \newcommand{\DecValTok}[1]{\textcolor[rgb]{0.25,0.63,0.44}{{#1}}}
    \newcommand{\BaseNTok}[1]{\textcolor[rgb]{0.25,0.63,0.44}{{#1}}}
    \newcommand{\FloatTok}[1]{\textcolor[rgb]{0.25,0.63,0.44}{{#1}}}
    \newcommand{\CharTok}[1]{\textcolor[rgb]{0.25,0.44,0.63}{{#1}}}
    \newcommand{\StringTok}[1]{\textcolor[rgb]{0.25,0.44,0.63}{{#1}}}
    \newcommand{\CommentTok}[1]{\textcolor[rgb]{0.38,0.63,0.69}{\textit{{#1}}}}
    \newcommand{\OtherTok}[1]{\textcolor[rgb]{0.00,0.44,0.13}{{#1}}}
    \newcommand{\AlertTok}[1]{\textcolor[rgb]{1.00,0.00,0.00}{\textbf{{#1}}}}
    \newcommand{\FunctionTok}[1]{\textcolor[rgb]{0.02,0.16,0.49}{{#1}}}
    \newcommand{\RegionMarkerTok}[1]{{#1}}
    \newcommand{\ErrorTok}[1]{\textcolor[rgb]{1.00,0.00,0.00}{\textbf{{#1}}}}
    \newcommand{\NormalTok}[1]{{#1}}
    
    % Additional commands for more recent versions of Pandoc
    \newcommand{\ConstantTok}[1]{\textcolor[rgb]{0.53,0.00,0.00}{{#1}}}
    \newcommand{\SpecialCharTok}[1]{\textcolor[rgb]{0.25,0.44,0.63}{{#1}}}
    \newcommand{\VerbatimStringTok}[1]{\textcolor[rgb]{0.25,0.44,0.63}{{#1}}}
    \newcommand{\SpecialStringTok}[1]{\textcolor[rgb]{0.73,0.40,0.53}{{#1}}}
    \newcommand{\ImportTok}[1]{{#1}}
    \newcommand{\DocumentationTok}[1]{\textcolor[rgb]{0.73,0.13,0.13}{\textit{{#1}}}}
    \newcommand{\AnnotationTok}[1]{\textcolor[rgb]{0.38,0.63,0.69}{\textbf{\textit{{#1}}}}}
    \newcommand{\CommentVarTok}[1]{\textcolor[rgb]{0.38,0.63,0.69}{\textbf{\textit{{#1}}}}}
    \newcommand{\VariableTok}[1]{\textcolor[rgb]{0.10,0.09,0.49}{{#1}}}
    \newcommand{\ControlFlowTok}[1]{\textcolor[rgb]{0.00,0.44,0.13}{\textbf{{#1}}}}
    \newcommand{\OperatorTok}[1]{\textcolor[rgb]{0.40,0.40,0.40}{{#1}}}
    \newcommand{\BuiltInTok}[1]{{#1}}
    \newcommand{\ExtensionTok}[1]{{#1}}
    \newcommand{\PreprocessorTok}[1]{\textcolor[rgb]{0.74,0.48,0.00}{{#1}}}
    \newcommand{\AttributeTok}[1]{\textcolor[rgb]{0.49,0.56,0.16}{{#1}}}
    \newcommand{\InformationTok}[1]{\textcolor[rgb]{0.38,0.63,0.69}{\textbf{\textit{{#1}}}}}
    \newcommand{\WarningTok}[1]{\textcolor[rgb]{0.38,0.63,0.69}{\textbf{\textit{{#1}}}}}
    
    
    % Define a nice break command that doesn't care if a line doesn't already
    % exist.
    \def\br{\hspace*{\fill} \\* }
    % Math Jax compatability definitions
    \def\gt{>}
    \def\lt{<}
    % Document parameters
    \title{index}
    
    
    

    % Pygments definitions
    
\makeatletter
\def\PY@reset{\let\PY@it=\relax \let\PY@bf=\relax%
    \let\PY@ul=\relax \let\PY@tc=\relax%
    \let\PY@bc=\relax \let\PY@ff=\relax}
\def\PY@tok#1{\csname PY@tok@#1\endcsname}
\def\PY@toks#1+{\ifx\relax#1\empty\else%
    \PY@tok{#1}\expandafter\PY@toks\fi}
\def\PY@do#1{\PY@bc{\PY@tc{\PY@ul{%
    \PY@it{\PY@bf{\PY@ff{#1}}}}}}}
\def\PY#1#2{\PY@reset\PY@toks#1+\relax+\PY@do{#2}}

\expandafter\def\csname PY@tok@w\endcsname{\def\PY@tc##1{\textcolor[rgb]{0.73,0.73,0.73}{##1}}}
\expandafter\def\csname PY@tok@c\endcsname{\let\PY@it=\textit\def\PY@tc##1{\textcolor[rgb]{0.25,0.50,0.50}{##1}}}
\expandafter\def\csname PY@tok@cp\endcsname{\def\PY@tc##1{\textcolor[rgb]{0.74,0.48,0.00}{##1}}}
\expandafter\def\csname PY@tok@k\endcsname{\let\PY@bf=\textbf\def\PY@tc##1{\textcolor[rgb]{0.00,0.50,0.00}{##1}}}
\expandafter\def\csname PY@tok@kp\endcsname{\def\PY@tc##1{\textcolor[rgb]{0.00,0.50,0.00}{##1}}}
\expandafter\def\csname PY@tok@kt\endcsname{\def\PY@tc##1{\textcolor[rgb]{0.69,0.00,0.25}{##1}}}
\expandafter\def\csname PY@tok@o\endcsname{\def\PY@tc##1{\textcolor[rgb]{0.40,0.40,0.40}{##1}}}
\expandafter\def\csname PY@tok@ow\endcsname{\let\PY@bf=\textbf\def\PY@tc##1{\textcolor[rgb]{0.67,0.13,1.00}{##1}}}
\expandafter\def\csname PY@tok@nb\endcsname{\def\PY@tc##1{\textcolor[rgb]{0.00,0.50,0.00}{##1}}}
\expandafter\def\csname PY@tok@nf\endcsname{\def\PY@tc##1{\textcolor[rgb]{0.00,0.00,1.00}{##1}}}
\expandafter\def\csname PY@tok@nc\endcsname{\let\PY@bf=\textbf\def\PY@tc##1{\textcolor[rgb]{0.00,0.00,1.00}{##1}}}
\expandafter\def\csname PY@tok@nn\endcsname{\let\PY@bf=\textbf\def\PY@tc##1{\textcolor[rgb]{0.00,0.00,1.00}{##1}}}
\expandafter\def\csname PY@tok@ne\endcsname{\let\PY@bf=\textbf\def\PY@tc##1{\textcolor[rgb]{0.82,0.25,0.23}{##1}}}
\expandafter\def\csname PY@tok@nv\endcsname{\def\PY@tc##1{\textcolor[rgb]{0.10,0.09,0.49}{##1}}}
\expandafter\def\csname PY@tok@no\endcsname{\def\PY@tc##1{\textcolor[rgb]{0.53,0.00,0.00}{##1}}}
\expandafter\def\csname PY@tok@nl\endcsname{\def\PY@tc##1{\textcolor[rgb]{0.63,0.63,0.00}{##1}}}
\expandafter\def\csname PY@tok@ni\endcsname{\let\PY@bf=\textbf\def\PY@tc##1{\textcolor[rgb]{0.60,0.60,0.60}{##1}}}
\expandafter\def\csname PY@tok@na\endcsname{\def\PY@tc##1{\textcolor[rgb]{0.49,0.56,0.16}{##1}}}
\expandafter\def\csname PY@tok@nt\endcsname{\let\PY@bf=\textbf\def\PY@tc##1{\textcolor[rgb]{0.00,0.50,0.00}{##1}}}
\expandafter\def\csname PY@tok@nd\endcsname{\def\PY@tc##1{\textcolor[rgb]{0.67,0.13,1.00}{##1}}}
\expandafter\def\csname PY@tok@s\endcsname{\def\PY@tc##1{\textcolor[rgb]{0.73,0.13,0.13}{##1}}}
\expandafter\def\csname PY@tok@sd\endcsname{\let\PY@it=\textit\def\PY@tc##1{\textcolor[rgb]{0.73,0.13,0.13}{##1}}}
\expandafter\def\csname PY@tok@si\endcsname{\let\PY@bf=\textbf\def\PY@tc##1{\textcolor[rgb]{0.73,0.40,0.53}{##1}}}
\expandafter\def\csname PY@tok@se\endcsname{\let\PY@bf=\textbf\def\PY@tc##1{\textcolor[rgb]{0.73,0.40,0.13}{##1}}}
\expandafter\def\csname PY@tok@sr\endcsname{\def\PY@tc##1{\textcolor[rgb]{0.73,0.40,0.53}{##1}}}
\expandafter\def\csname PY@tok@ss\endcsname{\def\PY@tc##1{\textcolor[rgb]{0.10,0.09,0.49}{##1}}}
\expandafter\def\csname PY@tok@sx\endcsname{\def\PY@tc##1{\textcolor[rgb]{0.00,0.50,0.00}{##1}}}
\expandafter\def\csname PY@tok@m\endcsname{\def\PY@tc##1{\textcolor[rgb]{0.40,0.40,0.40}{##1}}}
\expandafter\def\csname PY@tok@gh\endcsname{\let\PY@bf=\textbf\def\PY@tc##1{\textcolor[rgb]{0.00,0.00,0.50}{##1}}}
\expandafter\def\csname PY@tok@gu\endcsname{\let\PY@bf=\textbf\def\PY@tc##1{\textcolor[rgb]{0.50,0.00,0.50}{##1}}}
\expandafter\def\csname PY@tok@gd\endcsname{\def\PY@tc##1{\textcolor[rgb]{0.63,0.00,0.00}{##1}}}
\expandafter\def\csname PY@tok@gi\endcsname{\def\PY@tc##1{\textcolor[rgb]{0.00,0.63,0.00}{##1}}}
\expandafter\def\csname PY@tok@gr\endcsname{\def\PY@tc##1{\textcolor[rgb]{1.00,0.00,0.00}{##1}}}
\expandafter\def\csname PY@tok@ge\endcsname{\let\PY@it=\textit}
\expandafter\def\csname PY@tok@gs\endcsname{\let\PY@bf=\textbf}
\expandafter\def\csname PY@tok@gp\endcsname{\let\PY@bf=\textbf\def\PY@tc##1{\textcolor[rgb]{0.00,0.00,0.50}{##1}}}
\expandafter\def\csname PY@tok@go\endcsname{\def\PY@tc##1{\textcolor[rgb]{0.53,0.53,0.53}{##1}}}
\expandafter\def\csname PY@tok@gt\endcsname{\def\PY@tc##1{\textcolor[rgb]{0.00,0.27,0.87}{##1}}}
\expandafter\def\csname PY@tok@err\endcsname{\def\PY@bc##1{\setlength{\fboxsep}{0pt}\fcolorbox[rgb]{1.00,0.00,0.00}{1,1,1}{\strut ##1}}}
\expandafter\def\csname PY@tok@kc\endcsname{\let\PY@bf=\textbf\def\PY@tc##1{\textcolor[rgb]{0.00,0.50,0.00}{##1}}}
\expandafter\def\csname PY@tok@kd\endcsname{\let\PY@bf=\textbf\def\PY@tc##1{\textcolor[rgb]{0.00,0.50,0.00}{##1}}}
\expandafter\def\csname PY@tok@kn\endcsname{\let\PY@bf=\textbf\def\PY@tc##1{\textcolor[rgb]{0.00,0.50,0.00}{##1}}}
\expandafter\def\csname PY@tok@kr\endcsname{\let\PY@bf=\textbf\def\PY@tc##1{\textcolor[rgb]{0.00,0.50,0.00}{##1}}}
\expandafter\def\csname PY@tok@bp\endcsname{\def\PY@tc##1{\textcolor[rgb]{0.00,0.50,0.00}{##1}}}
\expandafter\def\csname PY@tok@fm\endcsname{\def\PY@tc##1{\textcolor[rgb]{0.00,0.00,1.00}{##1}}}
\expandafter\def\csname PY@tok@vc\endcsname{\def\PY@tc##1{\textcolor[rgb]{0.10,0.09,0.49}{##1}}}
\expandafter\def\csname PY@tok@vg\endcsname{\def\PY@tc##1{\textcolor[rgb]{0.10,0.09,0.49}{##1}}}
\expandafter\def\csname PY@tok@vi\endcsname{\def\PY@tc##1{\textcolor[rgb]{0.10,0.09,0.49}{##1}}}
\expandafter\def\csname PY@tok@vm\endcsname{\def\PY@tc##1{\textcolor[rgb]{0.10,0.09,0.49}{##1}}}
\expandafter\def\csname PY@tok@sa\endcsname{\def\PY@tc##1{\textcolor[rgb]{0.73,0.13,0.13}{##1}}}
\expandafter\def\csname PY@tok@sb\endcsname{\def\PY@tc##1{\textcolor[rgb]{0.73,0.13,0.13}{##1}}}
\expandafter\def\csname PY@tok@sc\endcsname{\def\PY@tc##1{\textcolor[rgb]{0.73,0.13,0.13}{##1}}}
\expandafter\def\csname PY@tok@dl\endcsname{\def\PY@tc##1{\textcolor[rgb]{0.73,0.13,0.13}{##1}}}
\expandafter\def\csname PY@tok@s2\endcsname{\def\PY@tc##1{\textcolor[rgb]{0.73,0.13,0.13}{##1}}}
\expandafter\def\csname PY@tok@sh\endcsname{\def\PY@tc##1{\textcolor[rgb]{0.73,0.13,0.13}{##1}}}
\expandafter\def\csname PY@tok@s1\endcsname{\def\PY@tc##1{\textcolor[rgb]{0.73,0.13,0.13}{##1}}}
\expandafter\def\csname PY@tok@mb\endcsname{\def\PY@tc##1{\textcolor[rgb]{0.40,0.40,0.40}{##1}}}
\expandafter\def\csname PY@tok@mf\endcsname{\def\PY@tc##1{\textcolor[rgb]{0.40,0.40,0.40}{##1}}}
\expandafter\def\csname PY@tok@mh\endcsname{\def\PY@tc##1{\textcolor[rgb]{0.40,0.40,0.40}{##1}}}
\expandafter\def\csname PY@tok@mi\endcsname{\def\PY@tc##1{\textcolor[rgb]{0.40,0.40,0.40}{##1}}}
\expandafter\def\csname PY@tok@il\endcsname{\def\PY@tc##1{\textcolor[rgb]{0.40,0.40,0.40}{##1}}}
\expandafter\def\csname PY@tok@mo\endcsname{\def\PY@tc##1{\textcolor[rgb]{0.40,0.40,0.40}{##1}}}
\expandafter\def\csname PY@tok@ch\endcsname{\let\PY@it=\textit\def\PY@tc##1{\textcolor[rgb]{0.25,0.50,0.50}{##1}}}
\expandafter\def\csname PY@tok@cm\endcsname{\let\PY@it=\textit\def\PY@tc##1{\textcolor[rgb]{0.25,0.50,0.50}{##1}}}
\expandafter\def\csname PY@tok@cpf\endcsname{\let\PY@it=\textit\def\PY@tc##1{\textcolor[rgb]{0.25,0.50,0.50}{##1}}}
\expandafter\def\csname PY@tok@c1\endcsname{\let\PY@it=\textit\def\PY@tc##1{\textcolor[rgb]{0.25,0.50,0.50}{##1}}}
\expandafter\def\csname PY@tok@cs\endcsname{\let\PY@it=\textit\def\PY@tc##1{\textcolor[rgb]{0.25,0.50,0.50}{##1}}}

\def\PYZbs{\char`\\}
\def\PYZus{\char`\_}
\def\PYZob{\char`\{}
\def\PYZcb{\char`\}}
\def\PYZca{\char`\^}
\def\PYZam{\char`\&}
\def\PYZlt{\char`\<}
\def\PYZgt{\char`\>}
\def\PYZsh{\char`\#}
\def\PYZpc{\char`\%}
\def\PYZdl{\char`\$}
\def\PYZhy{\char`\-}
\def\PYZsq{\char`\'}
\def\PYZdq{\char`\"}
\def\PYZti{\char`\~}
% for compatibility with earlier versions
\def\PYZat{@}
\def\PYZlb{[}
\def\PYZrb{]}
\makeatother


    % Exact colors from NB
    \definecolor{incolor}{rgb}{0.0, 0.0, 0.5}
    \definecolor{outcolor}{rgb}{0.545, 0.0, 0.0}



    
    % Prevent overflowing lines due to hard-to-break entities
    \sloppy 
    % Setup hyperref package
    \hypersetup{
      breaklinks=true,  % so long urls are correctly broken across lines
      colorlinks=true,
      urlcolor=urlcolor,
      linkcolor=linkcolor,
      citecolor=citecolor,
      }
    % Slightly bigger margins than the latex defaults
    
    \geometry{verbose,tmargin=1in,bmargin=1in,lmargin=1in,rmargin=1in}
    
    

    \begin{document}
    
    
    \maketitle
    
    

    
    \section{Predicting Disease Spread}\label{predicting-disease-spread}

Juan Alvarez and Emily Qiao

    \begin{Verbatim}[commandchars=\\\{\}]
{\color{incolor}In [{\color{incolor}5}]:} \PY{k+kn}{import} \PY{n+nn}{pandas} \PY{k}{as} \PY{n+nn}{pd}
        \PY{k+kn}{import} \PY{n+nn}{numpy} \PY{k}{as} \PY{n+nn}{np}
        \PY{k+kn}{from} \PY{n+nn}{sklearn}\PY{n+nn}{.}\PY{n+nn}{model\PYZus{}selection} \PY{k}{import} \PY{n}{train\PYZus{}test\PYZus{}split}
        \PY{k+kn}{from} \PY{n+nn}{sklearn}\PY{n+nn}{.}\PY{n+nn}{neighbors} \PY{k}{import} \PY{n}{KNeighborsRegressor}
        \PY{k+kn}{import} \PY{n+nn}{matplotlib}\PY{n+nn}{.}\PY{n+nn}{pyplot} \PY{k}{as} \PY{n+nn}{plt}
        \PY{k+kn}{import} \PY{n+nn}{seaborn} \PY{k}{as} \PY{n+nn}{sns} \PY{c+c1}{\PYZsh{} for visualiation}
        \PY{k+kn}{from} \PY{n+nn}{scipy}\PY{n+nn}{.}\PY{n+nn}{stats} \PY{k}{import} \PY{n}{ttest\PYZus{}ind} \PY{c+c1}{\PYZsh{} t\PYZhy{}tests}
        \PY{k+kn}{import} \PY{n+nn}{altair} \PY{k}{as} \PY{n+nn}{alt}
        \PY{n}{alt}\PY{o}{.}\PY{n}{renderers}\PY{o}{.}\PY{n}{enable}\PY{p}{(}\PY{l+s+s1}{\PYZsq{}}\PY{l+s+s1}{notebook}\PY{l+s+s1}{\PYZsq{}}\PY{p}{)} \PY{c+c1}{\PYZsh{} enable altair rendering}
        \PY{k+kn}{import} \PY{n+nn}{statsmodels}\PY{n+nn}{.}\PY{n+nn}{formula}\PY{n+nn}{.}\PY{n+nn}{api} \PY{k}{as} \PY{n+nn}{smf}
        \PY{k+kn}{import} \PY{n+nn}{statsmodels}\PY{n+nn}{.}\PY{n+nn}{api} \PY{k}{as} \PY{n+nn}{sm}
        
        \PY{k+kn}{import} \PY{n+nn}{warnings}
        \PY{n}{warnings}\PY{o}{.}\PY{n}{filterwarnings}\PY{p}{(}\PY{l+s+s1}{\PYZsq{}}\PY{l+s+s1}{ignore}\PY{l+s+s1}{\PYZsq{}}\PY{p}{)}
\end{Verbatim}


    \begin{Verbatim}[commandchars=\\\{\}]
{\color{incolor}In [{\color{incolor}2}]:} \PY{n}{data} \PY{o}{=} \PY{n}{pd}\PY{o}{.}\PY{n}{read\PYZus{}csv}\PY{p}{(}\PY{l+s+s1}{\PYZsq{}}\PY{l+s+s1}{data/dengue\PYZus{}features\PYZus{}train.csv}\PY{l+s+s1}{\PYZsq{}}\PY{p}{)}
        \PY{n}{outcome} \PY{o}{=} \PY{n}{pd}\PY{o}{.}\PY{n}{read\PYZus{}csv}\PY{p}{(}\PY{l+s+s1}{\PYZsq{}}\PY{l+s+s1}{data/dengue\PYZus{}labels\PYZus{}train.csv}\PY{l+s+s1}{\PYZsq{}}\PY{p}{)}
        \PY{n}{total} \PY{o}{=} \PY{n}{data}
        \PY{n}{total}\PY{p}{[}\PY{l+s+s1}{\PYZsq{}}\PY{l+s+s1}{total\PYZus{}cases}\PY{l+s+s1}{\PYZsq{}}\PY{p}{]} \PY{o}{=} \PY{n}{outcome}\PY{p}{[}\PY{l+s+s1}{\PYZsq{}}\PY{l+s+s1}{total\PYZus{}cases}\PY{l+s+s1}{\PYZsq{}}\PY{p}{]}
        \PY{n}{total}\PY{o}{.}\PY{n}{fillna}\PY{p}{(}\PY{n}{method} \PY{o}{=} \PY{l+s+s1}{\PYZsq{}}\PY{l+s+s1}{ffill}\PY{l+s+s1}{\PYZsq{}}\PY{p}{,} \PY{n}{inplace} \PY{o}{=} \PY{k+kc}{True}\PY{p}{)}
        
        \PY{n}{sj} \PY{o}{=} \PY{n}{total}\PY{p}{[}\PY{n}{total}\PY{o}{.}\PY{n}{city}\PY{o}{==}\PY{l+s+s1}{\PYZsq{}}\PY{l+s+s1}{sj}\PY{l+s+s1}{\PYZsq{}}\PY{p}{]}
        \PY{n}{iq} \PY{o}{=} \PY{n}{total}\PY{p}{[}\PY{n}{total}\PY{o}{.}\PY{n}{city}\PY{o}{==}\PY{l+s+s1}{\PYZsq{}}\PY{l+s+s1}{iq}\PY{l+s+s1}{\PYZsq{}}\PY{p}{]}
\end{Verbatim}


    \subsection{Problem Overview}\label{problem-overview}

    The dengue fever, as the name suggests, is a mosquito-borne disease
caused by the dengue viruses. The symptoms of dengue fever include, as
listed on
\href{https://www.webmd.com/a-to-z-guides/dengue-fever-reference\#1-2}{WebMD}:

\begin{itemize}
\tightlist
\item
  sudden, high fever
\item
  severe headaches
\item
  pain behind the eyes
\item
  severe joint and muscle pain
\item
  fatigue
\item
  nausea
\item
  vomiting
\item
  skin rash
\item
  mild bleeding
\end{itemize}

The symptoms are sometimes mild and mistaken for flu. Serious problems
can develop, however, if there was lack of early recognition and prompt
supportive treatment.

The dengue fever has become a global problem since the 1950s and is now
a leading cause of illness and death in the tropics and subtropics.
According to \href{https://www.cdc.gov/dengue/index.html}{Centers of
Disease Control and Prevention (CDC)}, there are not yet any vaccines to
prevent dengue fever.

Dengue fever is spread through the bite of female mosquito (Aedes
aegypti) when the mosquito takes blood of a infected person and bites a
healthy person within a week. Based on information given on
\href{http://www.who.int/denguecontrol/faq/en/index1.html}{World Health
Organization's website}, dengue outbreaks can occur anytime as long as
there are mosquitoes alive. Yet regions with high humidity and high
temperature usually have more likehood of dengue outbreaks.

In this project, we explore the dengue fever dataset and try to predict
the number of dengue cases based on environmental variables. The dataset
contains different aspects of data for two cities: San Juan
(\texttt{sj}) and Iquitos (\texttt{iq}). Aside from date indicators
(\texttt{week\_start\_date}), NDVI indexes (\texttt{ndvi\_se},
\texttt{ndvi\_sw}, \texttt{ndvi\_ne}, \texttt{ndvi\_nw}) are also given,
which measure the density of vegetation on a patch of land and can be
used as an indon icator of agricultural drought. These indexes are
relevant because mosquitoes lay eggs in water and the eggs need water to
hatch as well. If a region is experiencing emtreme drought, chances of
mosquitoes surviving in that area becomes low, hence reducing the spread
rate of dengue fever. There is also detailed climate information
provided in the dataset, coming from three different sources.
\texttt{station\_max\_temp\_c}, \texttt{station\_min\_temp\_c},
\texttt{station\_avg\_temp\_c}, \texttt{station\_precip\_mm}, and
\texttt{station\_diur\_temp\_rng\_c} are measurements from National
Ocean and Atmospheric Administration's (NOAA) Global Historical
Climatology Network (GHCN) daily climate data weather station, which
shows data about \emph{maximum temperature}, \emph{minimun temperature},
\emph{average temperature}, \emph{total precipitation}, and
\emph{diurnal temperature range} corresondingly.
\texttt{precipitation\_amt\_mm} data is generated by PERSIANN-CDR, an
artifical neural network that provides daily rainfall estimates, and
represents the total precipitation of a given region at a given time.
And finally, the rest of the data come from the NCEP Climate Forecast
System Reanalysis (CFSR), which is known for its accuracy of estimation,
where \texttt{reanalysis\_sat\_precip\_amt\_mm} represents \emph{total
precipitation}, \texttt{reanalysis\_dew\_point\_temp\_k} for \emph{mean
dew point temperature}, \texttt{reanalysis\_air\_temp\_k} for \emph{mean
air temperature}, \texttt{reanalysis\_relative\_humidity\_percent} for
\emph{mean relative humidity},
\texttt{reanalysis\_specific\_humidity\_g\_per\_kg} for \emph{mean
specific humidity}, \texttt{reanalysis\_precip\_amt\_kg\_per\_m2} for
\emph{total precipitation}, \texttt{reanalysis\_max\_air\_temp\_k} for
\emph{maximum air temperature}, \texttt{reanalysis\_min\_air\_temp\_k}
for \emph{minimum air temperature}, \texttt{reanalysis\_avg\_temp\_k}
for \emph{average air temperature}, and finally
\texttt{reanalysis\_tdtr\_k} for \emph{diurnal temperature range}.
Information about temperature, precipitation, and humidity is largely
needed because, as explained earlier, mosquitoes live in warm
environment and rely on water to lay eggs and hatch. It is very likely
to have a positive relationship between warm temperature, high humidity,
and more dangue cases reported.

    \subsection{Data Preparation}\label{data-preparation}

    \subsubsection{Data Transformation}\label{data-transformation}

After we loaded the training data (\texttt{dengue\_features\_train.csv}
and \texttt{dengue\_labels\_train}), we combined these two data frames
together to create a new data frame with all the environmental features
and the total cases of dengue fever. We worked on the combined data
frame throughout the project because it has all the information we need
and is easier to analyze.

    \subsubsection{Missing Values}\label{missing-values}

In order to take care of the missing values, we used the \texttt{ffill}
method built in \texttt{pandas}'s \texttt{fillna} function to fill the
missing data with the valid value forward. We chose to use this approach
because it is simply and fairly intuitive.

    \subsubsection{New Variable}\label{new-variable}

We created a new variable called \texttt{lag\_max} which is generated
calculateind the differece between the maximum temperatures
(\texttt{station\_max\_temp\_c}) of two consective week, because we
wanted to know about how change of climate may affect the amount of
mosquitoes coming in (aka cases of dengue fever).

    \subsection{Exploratory Data Analysis}\label{exploratory-data-analysis}

    We asked several questions to better understand the dataset.

\subsubsection{What is the distribution of the number of cases of Dengue
each
week?}\label{what-is-the-distribution-of-the-number-of-cases-of-dengue-each-week}

Here is the overall distribution of San Juan and Iquitos over the year:

    \begin{Verbatim}[commandchars=\\\{\}]
{\color{incolor}In [{\color{incolor}3}]:} \PY{n}{plt}\PY{o}{.}\PY{n}{subplots}\PY{p}{(}\PY{n}{figsize}\PY{o}{=}\PY{p}{(}\PY{l+m+mi}{15}\PY{p}{,}\PY{l+m+mi}{5}\PY{p}{)}\PY{p}{)}
        \PY{n}{g} \PY{o}{=} \PY{n}{sns}\PY{o}{.}\PY{n}{barplot}\PY{p}{(}\PY{n}{x}\PY{o}{=}\PY{l+s+s2}{\PYZdq{}}\PY{l+s+s2}{weekofyear}\PY{l+s+s2}{\PYZdq{}}\PY{p}{,} \PY{n}{y}\PY{o}{=}\PY{l+s+s2}{\PYZdq{}}\PY{l+s+s2}{total\PYZus{}cases}\PY{l+s+s2}{\PYZdq{}}\PY{p}{,} \PY{n}{hue} \PY{o}{=} \PY{l+s+s2}{\PYZdq{}}\PY{l+s+s2}{city}\PY{l+s+s2}{\PYZdq{}}\PY{p}{,} \PY{n}{data}\PY{o}{=}\PY{n}{total}\PY{p}{,} \PY{n}{ci} \PY{o}{=} \PY{k+kc}{None}\PY{p}{)}
        \PY{n}{g}\PY{o}{.}\PY{n}{set}\PY{p}{(}\PY{n}{ylabel}\PY{o}{=}\PY{l+s+s1}{\PYZsq{}}\PY{l+s+s1}{total cases}\PY{l+s+s1}{\PYZsq{}}\PY{p}{,} \PY{n}{xlabel} \PY{o}{=} \PY{l+s+s1}{\PYZsq{}}\PY{l+s+s1}{week of year}\PY{l+s+s1}{\PYZsq{}}\PY{p}{)}
\end{Verbatim}


\begin{Verbatim}[commandchars=\\\{\}]
{\color{outcolor}Out[{\color{outcolor}3}]:} [Text(0,0.5,'total cases'), Text(0.5,0,'week of year')]
\end{Verbatim}
            
    \begin{center}
    \adjustimage{max size={0.9\linewidth}{0.9\paperheight}}{output_11_1.png}
    \end{center}
    { \hspace*{\fill} \\}
    
    It is very clear that the distribution pattern of the two cities are
totally different. In San Juan, dengue cases increase a lot during the
second half of the year and go down in the first half of the year. While
in Iquitos, the curve looks more flat. The dengue disease is most severe
at the end of the year (week 50), and then goes down immediately.

The difference is even more obvious if we align the graphs of the two
cities vertically:

    \begin{Verbatim}[commandchars=\\\{\}]
{\color{incolor}In [{\color{incolor}6}]:} \PY{n}{g} \PY{o}{=} \PY{n}{sns}\PY{o}{.}\PY{n}{FacetGrid}\PY{p}{(}\PY{n}{data} \PY{o}{=} \PY{n}{total}\PY{p}{,} \PY{n}{row}\PY{o}{=}\PY{l+s+s2}{\PYZdq{}}\PY{l+s+s2}{city}\PY{l+s+s2}{\PYZdq{}}\PY{p}{,} \PY{n}{aspect}\PY{o}{=}\PY{l+m+mi}{4}\PY{p}{)}
        \PY{n}{g}\PY{o}{.}\PY{n}{map}\PY{p}{(}\PY{n}{sns}\PY{o}{.}\PY{n}{barplot}\PY{p}{,} \PY{l+s+s2}{\PYZdq{}}\PY{l+s+s2}{weekofyear}\PY{l+s+s2}{\PYZdq{}}\PY{p}{,} \PY{l+s+s2}{\PYZdq{}}\PY{l+s+s2}{total\PYZus{}cases}\PY{l+s+s2}{\PYZdq{}}\PY{p}{,} \PY{n}{ci}\PY{o}{=}\PY{k+kc}{None}\PY{p}{)}\PY{p}{;}
        \PY{n}{g}\PY{o}{.}\PY{n}{set}\PY{p}{(}\PY{n}{ylabel}\PY{o}{=}\PY{l+s+s1}{\PYZsq{}}\PY{l+s+s1}{total cases}\PY{l+s+s1}{\PYZsq{}}\PY{p}{)}
\end{Verbatim}


\begin{Verbatim}[commandchars=\\\{\}]
{\color{outcolor}Out[{\color{outcolor}6}]:} <seaborn.axisgrid.FacetGrid at 0x1c17f68a58>
\end{Verbatim}
            
    \begin{center}
    \adjustimage{max size={0.9\linewidth}{0.9\paperheight}}{output_13_1.png}
    \end{center}
    { \hspace*{\fill} \\}
    
    What made the distrbution of the two cities so different? Interestingly,
\textbf{San Juan has a tropical monsoon climate}. Rainfall is
well-distributed throughout the year, but \textbf{January, February, and
March are the driest}. This explains why cases of dengue decrease during
the first quarter of the year (mosquitoes don't like dry weather) and
increase from week 17 (April). On the other hand, \textbf{Iquitos has
tropical rainforest climate}, where there is constant rainfall
throughout the year, \textbf{without a distinct dry season}. This is why
the curve for Iquitos is a lot more flat.

\subsubsection{How does the number of cases fluctuate over time? Do
these temporal relationships persist in both
locations?}\label{how-does-the-number-of-cases-fluctuate-over-time-do-these-temporal-relationships-persist-in-both-locations}

It turns out that \textbf{the number of cases actually fluctuates a lot
over time}. In San Juan, there were significantly more cases in 1994 and
1998, while in years like 2000, 2002 and 2008, it seems like dengue
fever was pretty under control.

    \begin{Verbatim}[commandchars=\\\{\}]
{\color{incolor}In [{\color{incolor}9}]:} \PY{n}{plt}\PY{o}{.}\PY{n}{subplots}\PY{p}{(}\PY{n}{figsize}\PY{o}{=}\PY{p}{(}\PY{l+m+mi}{15}\PY{p}{,}\PY{l+m+mi}{5}\PY{p}{)}\PY{p}{)}
        \PY{n}{g} \PY{o}{=} \PY{n}{sns}\PY{o}{.}\PY{n}{barplot}\PY{p}{(}\PY{n}{x}\PY{o}{=}\PY{l+s+s2}{\PYZdq{}}\PY{l+s+s2}{year}\PY{l+s+s2}{\PYZdq{}}\PY{p}{,} \PY{n}{y}\PY{o}{=}\PY{l+s+s2}{\PYZdq{}}\PY{l+s+s2}{total\PYZus{}cases}\PY{l+s+s2}{\PYZdq{}}\PY{p}{,} \PY{n}{data}\PY{o}{=}\PY{n}{sj}\PY{p}{,} \PY{n}{ci} \PY{o}{=} \PY{k+kc}{None}\PY{p}{,} \PY{n}{color} \PY{o}{=} \PY{l+s+s2}{\PYZdq{}}\PY{l+s+s2}{steelblue}\PY{l+s+s2}{\PYZdq{}}\PY{p}{)}
        \PY{n}{g}\PY{o}{.}\PY{n}{set}\PY{p}{(}\PY{n}{ylabel}\PY{o}{=}\PY{l+s+s1}{\PYZsq{}}\PY{l+s+s1}{total cases}\PY{l+s+s1}{\PYZsq{}}\PY{p}{,} \PY{n}{xlabel} \PY{o}{=} \PY{l+s+s1}{\PYZsq{}}\PY{l+s+s1}{year}\PY{l+s+s1}{\PYZsq{}}\PY{p}{,} \PY{n}{title} \PY{o}{=} \PY{l+s+s2}{\PYZdq{}}\PY{l+s+s2}{San Juan}\PY{l+s+s2}{\PYZdq{}}\PY{p}{)}
\end{Verbatim}


\begin{Verbatim}[commandchars=\\\{\}]
{\color{outcolor}Out[{\color{outcolor}9}]:} [Text(0,0.5,'total cases'), Text(0.5,0,'year'), Text(0.5,1,'San Juan')]
\end{Verbatim}
            
    \begin{center}
    \adjustimage{max size={0.9\linewidth}{0.9\paperheight}}{output_15_1.png}
    \end{center}
    { \hspace*{\fill} \\}
    
    And in Iquitos, there was almost no cases of dengue fever before 2002. A
outbreak happened in 2002, although the number of cases is still a lot
smaller than San Juan's. The disease got controlled in 2003, but became
resurgent in 2004. Then there was another outbreak in 2008.

    \begin{Verbatim}[commandchars=\\\{\}]
{\color{incolor}In [{\color{incolor}8}]:} \PY{n}{plt}\PY{o}{.}\PY{n}{subplots}\PY{p}{(}\PY{n}{figsize}\PY{o}{=}\PY{p}{(}\PY{l+m+mi}{15}\PY{p}{,}\PY{l+m+mi}{5}\PY{p}{)}\PY{p}{)}
        \PY{n}{g} \PY{o}{=} \PY{n}{sns}\PY{o}{.}\PY{n}{barplot}\PY{p}{(}\PY{n}{x}\PY{o}{=}\PY{l+s+s2}{\PYZdq{}}\PY{l+s+s2}{year}\PY{l+s+s2}{\PYZdq{}}\PY{p}{,} \PY{n}{y}\PY{o}{=}\PY{l+s+s2}{\PYZdq{}}\PY{l+s+s2}{total\PYZus{}cases}\PY{l+s+s2}{\PYZdq{}}\PY{p}{,} \PY{n}{data}\PY{o}{=}\PY{n}{iq}\PY{p}{,} \PY{n}{ci} \PY{o}{=} \PY{k+kc}{None}\PY{p}{,} \PY{n}{color} \PY{o}{=} \PY{l+s+s2}{\PYZdq{}}\PY{l+s+s2}{steelblue}\PY{l+s+s2}{\PYZdq{}}\PY{p}{)}
        \PY{n}{g}\PY{o}{.}\PY{n}{set}\PY{p}{(}\PY{n}{ylabel}\PY{o}{=}\PY{l+s+s1}{\PYZsq{}}\PY{l+s+s1}{total cases}\PY{l+s+s1}{\PYZsq{}}\PY{p}{,} \PY{n}{xlabel} \PY{o}{=} \PY{l+s+s1}{\PYZsq{}}\PY{l+s+s1}{year}\PY{l+s+s1}{\PYZsq{}}\PY{p}{,} \PY{n}{title} \PY{o}{=} \PY{l+s+s2}{\PYZdq{}}\PY{l+s+s2}{Iquitos}\PY{l+s+s2}{\PYZdq{}}\PY{p}{)}
\end{Verbatim}


\begin{Verbatim}[commandchars=\\\{\}]
{\color{outcolor}Out[{\color{outcolor}8}]:} [Text(0,0.5,'total cases'), Text(0.5,0,'year'), Text(0.5,1,'Iquitos')]
\end{Verbatim}
            
    \begin{center}
    \adjustimage{max size={0.9\linewidth}{0.9\paperheight}}{output_17_1.png}
    \end{center}
    { \hspace*{\fill} \\}
    
    To look at the two graphs together, we can see that the way that
\textbf{the number of cases of Dengue fever varies is independent of the
city} we are looking at. Temporal relationships do not persist in both
locations. We see that in Iquitos there was a spike in cases from 2003
to 2004 and 2005 to 2006, but there was a sharp decrease in cases in San
Juan.

    \subsection{Which variables are correlated with your outcome of interest
(total\_cases)? Are these correlations consistent in both cities (you
may want to calculate
this)?}\label{which-variables-are-correlated-with-your-outcome-of-interest-total_cases-are-these-correlations-consistent-in-both-cities-you-may-want-to-calculate-this}

Lets make bar charts(representing each city), and see which variables
correlate to total\_cases by looking at thier distribution.

    Visualizations for the varaibles can be found on \textbf{vis.ipynb}. For
a Negative Binomial Regression model, we want to pick variables that
follow a \emph{negative binomial distribution}. We see that depending on
the city we are looking at, there are some varaibles that follow this
curve. The variances within each variable have to be higher than the
means within each varaible. These differences suggest that
over-dispersion is present and that a Negative Binomial model would be
appropriate.

    \subsection{Statistical Modeling}\label{statistical-modeling}

    Looking at the previous section about variables correlated to our model,
our team decided that a Negative Binomial Regression model was the best
fit for our distribution. As we can see in \textbf{vis.ipynb}, the
distribution of \emph{total\_cases} follows a negative binomial
distribution. Since the mean and variance of the data are different in
terms of a relative threshold, we can not pick a Poisson model. The
variance is higher than the mean in our case, which suggests
over-dispersion. The negative binomial distribution has one more
parameter in its function than the Poisson regression, which adjusts the
variance independently from the mean. Our outcome (\emph{total\_cases})
is a discrete count, so we can use this model.

    \subsubsection{What variables to pick for the
model?}\label{what-variables-to-pick-for-the-model}

    Lets take a quick look at our data, before we see any visualizations.

    \begin{Verbatim}[commandchars=\\\{\}]
{\color{incolor}In [{\color{incolor}8}]:} \PY{n}{total}\PY{o}{.}\PY{n}{total\PYZus{}cases}\PY{o}{.}\PY{n}{describe}\PY{p}{(}\PY{p}{)}
\end{Verbatim}


\begin{Verbatim}[commandchars=\\\{\}]
{\color{outcolor}Out[{\color{outcolor}8}]:} count    1456.000000
        mean       24.675137
        std        43.596000
        min         0.000000
        25\%         5.000000
        50\%        12.000000
        75\%        28.000000
        max       461.000000
        Name: total\_cases, dtype: float64
\end{Verbatim}
            
    \begin{Verbatim}[commandchars=\\\{\}]
{\color{incolor}In [{\color{incolor}9}]:} \PY{n}{total}\PY{o}{.}\PY{n}{describe}\PY{p}{(}\PY{p}{)}
\end{Verbatim}


\begin{Verbatim}[commandchars=\\\{\}]
{\color{outcolor}Out[{\color{outcolor}9}]:}               year   weekofyear      ndvi\_ne      ndvi\_nw      ndvi\_se  \textbackslash{}
        count  1456.000000  1456.000000  1456.000000  1456.000000  1456.000000   
        mean   2001.031593    26.503434     0.131271     0.128068     0.202606   
        std       5.408314    15.019437     0.138527     0.119561     0.074409   
        min    1990.000000     1.000000    -0.406250    -0.456100    -0.015533   
        25\%    1997.000000    13.750000     0.039100     0.048250     0.152795   
        50\%    2002.000000    26.500000     0.113900     0.115926     0.195664   
        75\%    2005.000000    39.250000     0.232018     0.213429     0.247461   
        max    2010.000000    53.000000     0.508357     0.454429     0.538314   
        
                   ndvi\_sw  precipitation\_amt\_mm  reanalysis\_air\_temp\_k  \textbackslash{}
        count  1456.000000           1456.000000            1456.000000   
        mean      0.201987             45.686937             298.697965   
        std       0.083592             43.779318               1.361950   
        min      -0.063457              0.000000             294.635714   
        25\%       0.144455              9.655000             297.654643   
        50\%       0.190121             38.235000             298.646429   
        75\%       0.246775             70.227500             299.827500   
        max       0.546017            390.600000             302.200000   
        
               reanalysis\_avg\_temp\_k  reanalysis\_dew\_point\_temp\_k     {\ldots}       \textbackslash{}
        count            1456.000000                  1456.000000     {\ldots}        
        mean              299.221483                   295.245445     {\ldots}        
        std                 1.262676                     1.527527     {\ldots}        
        min               294.892857                   289.642857     {\ldots}        
        25\%               298.257143                   294.119643     {\ldots}        
        50\%               299.285714                   295.639286     {\ldots}        
        75\%               300.207143                   296.460000     {\ldots}        
        max               302.928571                   298.450000     {\ldots}        
        
               reanalysis\_relative\_humidity\_percent  reanalysis\_sat\_precip\_amt\_mm  \textbackslash{}
        count                           1456.000000                   1456.000000   
        mean                              82.176203                     45.686937   
        std                                7.161016                     43.779318   
        min                               57.787143                      0.000000   
        25\%                               77.185714                      9.655000   
        50\%                               80.301429                     38.235000   
        75\%                               86.406429                     70.227500   
        max                               98.610000                    390.600000   
        
               reanalysis\_specific\_humidity\_g\_per\_kg  reanalysis\_tdtr\_k  \textbackslash{}
        count                            1456.000000        1456.000000   
        mean                               16.745565           4.898656   
        std                                 1.542276           3.542340   
        min                                11.715714           1.357143   
        25\%                                15.560000           2.328571   
        50\%                                17.087143           2.857143   
        75\%                                17.977500           7.617857   
        max                                20.461429          16.028571   
        
               station\_avg\_temp\_c  station\_diur\_temp\_rng\_c  station\_max\_temp\_c  \textbackslash{}
        count         1456.000000              1456.000000         1456.000000   
        mean            27.180313                 8.085646           32.443338   
        std              1.280861                 2.122836            1.960769   
        min             21.400000                 4.528571           26.700000   
        25\%             26.300000                 6.528571           31.100000   
        50\%             27.400000                 7.364286           32.800000   
        75\%             28.132143                 9.600000           33.900000   
        max             30.800000                15.800000           42.200000   
        
               station\_min\_temp\_c  station\_precip\_mm  total\_cases  
        count         1456.000000        1456.000000  1456.000000  
        mean            22.099863          39.194162    24.675137  
        std              1.569113          47.363305    43.596000  
        min             14.700000           0.000000     0.000000  
        25\%             21.100000           8.700000     5.000000  
        50\%             22.200000          23.800000    12.000000  
        75\%             23.300000          53.675000    28.000000  
        max             25.600000         543.300000   461.000000  
        
        [8 rows x 23 columns]
\end{Verbatim}
            
    \subsubsection{We need to select the columns where there is
overdispersion
present.}\label{we-need-to-select-the-columns-where-there-is-overdispersion-present.}

Using a simple for-loop, we can filter out what columns have a larger
variance than mean.

    \begin{Verbatim}[commandchars=\\\{\}]
{\color{incolor}In [{\color{incolor}10}]:} \PY{k}{for} \PY{n}{column} \PY{o+ow}{in} \PY{n}{sj}\PY{p}{:}
             \PY{k}{if} \PY{o+ow}{not} \PY{p}{(}\PY{n}{column} \PY{o}{==} \PY{l+s+s1}{\PYZsq{}}\PY{l+s+s1}{city}\PY{l+s+s1}{\PYZsq{}} \PY{o+ow}{or} \PY{n}{column} \PY{o}{==} \PY{l+s+s1}{\PYZsq{}}\PY{l+s+s1}{week\PYZus{}start\PYZus{}date}\PY{l+s+s1}{\PYZsq{}}\PY{p}{)}\PY{p}{:}
                 \PY{k}{if}\PY{p}{(}\PY{n}{sj}\PY{p}{[}\PY{n}{column}\PY{p}{]}\PY{o}{.}\PY{n}{mean}\PY{p}{(}\PY{p}{)} \PY{o}{\PYZlt{}} \PY{n}{sj}\PY{p}{[}\PY{n}{column}\PY{p}{]}\PY{o}{.}\PY{n}{var}\PY{p}{(}\PY{p}{)}\PY{p}{)}\PY{p}{:}
                     \PY{n+nb}{print}\PY{p}{(}\PY{n}{column}\PY{p}{)}
\end{Verbatim}


    \begin{Verbatim}[commandchars=\\\{\}]
weekofyear
precipitation\_amt\_mm
reanalysis\_precip\_amt\_kg\_per\_m2
reanalysis\_sat\_precip\_amt\_mm
station\_precip\_mm
total\_cases

    \end{Verbatim}

    \begin{Verbatim}[commandchars=\\\{\}]
{\color{incolor}In [{\color{incolor}11}]:} \PY{k}{for} \PY{n}{column} \PY{o+ow}{in} \PY{n}{iq}\PY{p}{:}
             \PY{k}{if} \PY{o+ow}{not} \PY{p}{(}\PY{n}{column} \PY{o}{==} \PY{l+s+s1}{\PYZsq{}}\PY{l+s+s1}{city}\PY{l+s+s1}{\PYZsq{}} \PY{o+ow}{or} \PY{n}{column} \PY{o}{==} \PY{l+s+s1}{\PYZsq{}}\PY{l+s+s1}{week\PYZus{}start\PYZus{}date}\PY{l+s+s1}{\PYZsq{}}\PY{p}{)}\PY{p}{:}
                 \PY{k}{if}\PY{p}{(}\PY{n}{iq}\PY{p}{[}\PY{n}{column}\PY{p}{]}\PY{o}{.}\PY{n}{mean}\PY{p}{(}\PY{p}{)} \PY{o}{\PYZlt{}} \PY{n}{iq}\PY{p}{[}\PY{n}{column}\PY{p}{]}\PY{o}{.}\PY{n}{var}\PY{p}{(}\PY{p}{)}\PY{p}{)}\PY{p}{:}
                     \PY{n+nb}{print}\PY{p}{(}\PY{n}{column}\PY{p}{)}
\end{Verbatim}


    \begin{Verbatim}[commandchars=\\\{\}]
weekofyear
precipitation\_amt\_mm
reanalysis\_precip\_amt\_kg\_per\_m2
reanalysis\_sat\_precip\_amt\_mm
station\_precip\_mm
total\_cases

    \end{Verbatim}

    List of variables that follow NB distribution for San Juan: -
station\_precip\_mm - precipitation\_amt\_mm -
reanalysis\_sat\_precip\_amt\_mm - reanalysis\_precip\_amt\_kg\_per\_m2

List of variables that follow NB distribution for Iquitos: -
station\_precip\_mm - precipitation\_amt\_mm -
reanalysis\_sat\_precip\_amt\_mm - reanalysis\_precip\_amt\_kg\_per\_m2

    \begin{Verbatim}[commandchars=\\\{\}]
{\color{incolor}In [{\color{incolor}27}]:} \PY{n}{sj\PYZus{}model} \PY{o}{=} \PY{n}{smf}\PY{o}{.}\PY{n}{glm}\PY{p}{(}\PY{n}{formula}\PY{o}{=}\PY{l+s+s1}{\PYZsq{}}\PY{l+s+s1}{total\PYZus{}cases \PYZti{} reanalysis\PYZus{}precip\PYZus{}amt\PYZus{}kg\PYZus{}per\PYZus{}m2}\PY{l+s+s1}{\PYZsq{}}\PY{p}{,} \PY{n}{data}\PY{o}{=}\PY{n}{sj}\PY{p}{,} \PY{n}{family}\PY{o}{=}\PY{n}{sm}\PY{o}{.}\PY{n}{families}\PY{o}{.}\PY{n}{NegativeBinomial}\PY{p}{(}\PY{p}{)}\PY{p}{)}\PY{o}{.}\PY{n}{fit}\PY{p}{(}\PY{p}{)}
         \PY{n}{iq\PYZus{}model} \PY{o}{=} \PY{n}{smf}\PY{o}{.}\PY{n}{glm}\PY{p}{(}\PY{n}{formula}\PY{o}{=}\PY{l+s+s1}{\PYZsq{}}\PY{l+s+s1}{total\PYZus{}cases \PYZti{} reanalysis\PYZus{}precip\PYZus{}amt\PYZus{}kg\PYZus{}per\PYZus{}m2}\PY{l+s+s1}{\PYZsq{}}\PY{p}{,} \PY{n}{data}\PY{o}{=}\PY{n}{iq}\PY{p}{,} \PY{n}{family}\PY{o}{=}\PY{n}{sm}\PY{o}{.}\PY{n}{families}\PY{o}{.}\PY{n}{NegativeBinomial}\PY{p}{(}\PY{p}{)}\PY{p}{)}\PY{o}{.}\PY{n}{fit}\PY{p}{(}\PY{p}{)}
\end{Verbatim}


    \subsubsection{San Juan Model}\label{san-juan-model}

    \begin{Verbatim}[commandchars=\\\{\}]
{\color{incolor}In [{\color{incolor}28}]:} \PY{n+nb}{print}\PY{p}{(}\PY{n}{sj\PYZus{}model}\PY{o}{.}\PY{n}{summary}\PY{p}{(}\PY{p}{)}\PY{p}{)}
\end{Verbatim}


    \begin{Verbatim}[commandchars=\\\{\}]
                 Generalized Linear Model Regression Results                  
==============================================================================
Dep. Variable:            total\_cases   No. Observations:                  936
Model:                            GLM   Df Residuals:                      934
Model Family:        NegativeBinomial   Df Model:                            1
Link Function:                    log   Scale:                          1.0000
Method:                          IRLS   Log-Likelihood:                -4240.8
Date:                Tue, 13 Nov 2018   Deviance:                       1075.8
Time:                        00:17:05   Pearson chi2:                 2.00e+03
No. Iterations:                    13   Covariance Type:             nonrobust
===================================================================================================
                                      coef    std err          z      P>|z|      [0.025      0.975]
---------------------------------------------------------------------------------------------------
Intercept                           3.3294      0.044     76.313      0.000       3.244       3.415
reanalysis\_precip\_amt\_kg\_per\_m2     0.0061      0.001      6.614      0.000       0.004       0.008
===================================================================================================

    \end{Verbatim}

    \subsubsection{Iquitos Model}\label{iquitos-model}

    \begin{Verbatim}[commandchars=\\\{\}]
{\color{incolor}In [{\color{incolor}29}]:} \PY{n+nb}{print}\PY{p}{(}\PY{n}{iq\PYZus{}model}\PY{o}{.}\PY{n}{summary}\PY{p}{(}\PY{p}{)}\PY{p}{)}
\end{Verbatim}


    \begin{Verbatim}[commandchars=\\\{\}]
                 Generalized Linear Model Regression Results                  
==============================================================================
Dep. Variable:            total\_cases   No. Observations:                  520
Model:                            GLM   Df Residuals:                      518
Model Family:        NegativeBinomial   Df Model:                            1
Link Function:                    log   Scale:                          1.0000
Method:                          IRLS   Log-Likelihood:                -1599.2
Date:                Tue, 13 Nov 2018   Deviance:                       775.04
Time:                        00:17:08   Pearson chi2:                     896.
No. Iterations:                    10   Covariance Type:             nonrobust
===================================================================================================
                                      coef    std err          z      P>|z|      [0.025      0.975]
---------------------------------------------------------------------------------------------------
Intercept                           1.7908      0.071     25.141      0.000       1.651       1.930
reanalysis\_precip\_amt\_kg\_per\_m2     0.0038      0.001      4.157      0.000       0.002       0.006
===================================================================================================

    \end{Verbatim}

    Our models show that the only type of variable that fits the model is
related to precipitation. The p-value of the variable is below 0.05, so
we can reject the null hypothesis and say that there is a significant
correlation between \emph{total\_cases} and
\emph{reanalysis\_precip\_amt\_kg\_per\_m2}

    \subsubsection{Make Predictions Given
Model}\label{make-predictions-given-model}

    \subsubsection{Plot Iquitos Predictions}\label{plot-iquitos-predictions}

    \begin{Verbatim}[commandchars=\\\{\}]
{\color{incolor}In [{\color{incolor}31}]:} \PY{n}{plt}\PY{o}{.}\PY{n}{figure}\PY{p}{(}\PY{n}{figsize}\PY{o}{=}\PY{p}{(}\PY{l+m+mi}{20}\PY{p}{,}\PY{l+m+mi}{10}\PY{p}{)}\PY{p}{)}
         \PY{n}{plt}\PY{o}{.}\PY{n}{scatter}\PY{p}{(}\PY{n}{x}\PY{o}{=}\PY{n}{iq}\PY{p}{[}\PY{l+s+s1}{\PYZsq{}}\PY{l+s+s1}{weekofyear}\PY{l+s+s1}{\PYZsq{}}\PY{p}{]}\PY{p}{,} \PY{n}{y}\PY{o}{=}\PY{n}{iq}\PY{p}{[}\PY{l+s+s1}{\PYZsq{}}\PY{l+s+s1}{total\PYZus{}cases}\PY{l+s+s1}{\PYZsq{}}\PY{p}{]}\PY{p}{,} \PY{n}{c}\PY{o}{=}\PY{l+s+s2}{\PYZdq{}}\PY{l+s+s2}{\PYZsh{}85754d}\PY{l+s+s2}{\PYZdq{}}\PY{p}{,} \PY{n}{label}\PY{o}{=}\PY{l+s+s2}{\PYZdq{}}\PY{l+s+s2}{actual cases}\PY{l+s+s2}{\PYZdq{}}\PY{p}{)}
         \PY{n}{plt}\PY{o}{.}\PY{n}{scatter}\PY{p}{(}\PY{n}{x}\PY{o}{=}\PY{n}{iq}\PY{p}{[}\PY{l+s+s1}{\PYZsq{}}\PY{l+s+s1}{weekofyear}\PY{l+s+s1}{\PYZsq{}}\PY{p}{]}\PY{p}{,} \PY{n}{y}\PY{o}{=}\PY{n}{iq}\PY{p}{[}\PY{l+s+s1}{\PYZsq{}}\PY{l+s+s1}{iq\PYZus{}predictions}\PY{l+s+s1}{\PYZsq{}}\PY{p}{]}\PY{p}{,} \PY{n}{c}\PY{o}{=}\PY{l+s+s2}{\PYZdq{}}\PY{l+s+s2}{\PYZsh{}4b2e83}\PY{l+s+s2}{\PYZdq{}}\PY{p}{,} \PY{n}{label}\PY{o}{=}\PY{l+s+s2}{\PYZdq{}}\PY{l+s+s2}{predicted cases}\PY{l+s+s2}{\PYZdq{}}\PY{p}{,} \PY{n}{alpha} \PY{o}{=} \PY{l+m+mf}{0.6}\PY{p}{)}
         \PY{n}{plt}\PY{o}{.}\PY{n}{legend}\PY{p}{(}\PY{n}{loc}\PY{o}{=}\PY{l+s+s1}{\PYZsq{}}\PY{l+s+s1}{upper left}\PY{l+s+s1}{\PYZsq{}}\PY{p}{)}
\end{Verbatim}


\begin{Verbatim}[commandchars=\\\{\}]
{\color{outcolor}Out[{\color{outcolor}31}]:} <matplotlib.legend.Legend at 0x1c2165e6d8>
\end{Verbatim}
            
    \begin{center}
    \adjustimage{max size={0.9\linewidth}{0.9\paperheight}}{output_39_1.png}
    \end{center}
    { \hspace*{\fill} \\}
    
    \subsubsection{Plot San Juan
Predictions}\label{plot-san-juan-predictions}

    \begin{Verbatim}[commandchars=\\\{\}]
{\color{incolor}In [{\color{incolor}32}]:} \PY{n}{plt}\PY{o}{.}\PY{n}{figure}\PY{p}{(}\PY{n}{figsize}\PY{o}{=}\PY{p}{(}\PY{l+m+mi}{20}\PY{p}{,}\PY{l+m+mi}{10}\PY{p}{)}\PY{p}{)}
         \PY{n}{plt}\PY{o}{.}\PY{n}{scatter}\PY{p}{(}\PY{n}{x}\PY{o}{=}\PY{n}{sj}\PY{p}{[}\PY{l+s+s1}{\PYZsq{}}\PY{l+s+s1}{weekofyear}\PY{l+s+s1}{\PYZsq{}}\PY{p}{]}\PY{p}{,} \PY{n}{y}\PY{o}{=}\PY{n}{sj}\PY{p}{[}\PY{l+s+s1}{\PYZsq{}}\PY{l+s+s1}{total\PYZus{}cases}\PY{l+s+s1}{\PYZsq{}}\PY{p}{]}\PY{p}{,} \PY{n}{c}\PY{o}{=}\PY{l+s+s2}{\PYZdq{}}\PY{l+s+s2}{\PYZsh{}85754d}\PY{l+s+s2}{\PYZdq{}}\PY{p}{,} \PY{n}{label}\PY{o}{=}\PY{l+s+s2}{\PYZdq{}}\PY{l+s+s2}{actual cases}\PY{l+s+s2}{\PYZdq{}}\PY{p}{)}
         \PY{n}{plt}\PY{o}{.}\PY{n}{scatter}\PY{p}{(}\PY{n}{x}\PY{o}{=}\PY{n}{sj}\PY{p}{[}\PY{l+s+s1}{\PYZsq{}}\PY{l+s+s1}{weekofyear}\PY{l+s+s1}{\PYZsq{}}\PY{p}{]}\PY{p}{,} \PY{n}{y}\PY{o}{=}\PY{n}{sj}\PY{p}{[}\PY{l+s+s1}{\PYZsq{}}\PY{l+s+s1}{sj\PYZus{}predictions}\PY{l+s+s1}{\PYZsq{}}\PY{p}{]}\PY{p}{,} \PY{n}{c}\PY{o}{=}\PY{l+s+s2}{\PYZdq{}}\PY{l+s+s2}{\PYZsh{}4b2e83}\PY{l+s+s2}{\PYZdq{}}\PY{p}{,} \PY{n}{label}\PY{o}{=}\PY{l+s+s2}{\PYZdq{}}\PY{l+s+s2}{predicted cases}\PY{l+s+s2}{\PYZdq{}}\PY{p}{,} \PY{n}{alpha} \PY{o}{=} \PY{l+m+mf}{0.6}\PY{p}{)}
         \PY{n}{plt}\PY{o}{.}\PY{n}{legend}\PY{p}{(}\PY{n}{loc}\PY{o}{=}\PY{l+s+s1}{\PYZsq{}}\PY{l+s+s1}{upper left}\PY{l+s+s1}{\PYZsq{}}\PY{p}{)}
\end{Verbatim}


\begin{Verbatim}[commandchars=\\\{\}]
{\color{outcolor}Out[{\color{outcolor}32}]:} <matplotlib.legend.Legend at 0x1c2139a748>
\end{Verbatim}
            
    \begin{center}
    \adjustimage{max size={0.9\linewidth}{0.9\paperheight}}{output_41_1.png}
    \end{center}
    { \hspace*{\fill} \\}
    
    \subsubsection{Load Test Data}\label{load-test-data}

    \begin{Verbatim}[commandchars=\\\{\}]
{\color{incolor}In [{\color{incolor}34}]:} \PY{n}{test\PYZus{}data} \PY{o}{=} \PY{n}{pd}\PY{o}{.}\PY{n}{read\PYZus{}csv}\PY{p}{(}\PY{l+s+s1}{\PYZsq{}}\PY{l+s+s1}{data/dengue\PYZus{}features\PYZus{}test.csv}\PY{l+s+s1}{\PYZsq{}}\PY{p}{)}
         \PY{n}{test\PYZus{}sj} \PY{o}{=} \PY{n}{test\PYZus{}data}\PY{p}{[}\PY{n}{test\PYZus{}data}\PY{o}{.}\PY{n}{city} \PY{o}{==} \PY{l+s+s1}{\PYZsq{}}\PY{l+s+s1}{sj}\PY{l+s+s1}{\PYZsq{}}\PY{p}{]}\PY{o}{.}\PY{n}{copy}\PY{p}{(}\PY{p}{)}
         \PY{n}{test\PYZus{}sj}\PY{o}{.}\PY{n}{fillna}\PY{p}{(}\PY{n}{method} \PY{o}{=} \PY{l+s+s1}{\PYZsq{}}\PY{l+s+s1}{ffill}\PY{l+s+s1}{\PYZsq{}}\PY{p}{,} \PY{n}{inplace} \PY{o}{=} \PY{k+kc}{True}\PY{p}{)}
         \PY{n}{predictions\PYZus{}sj} \PY{o}{=} \PY{n}{sj\PYZus{}model}\PY{o}{.}\PY{n}{predict}\PY{p}{(}\PY{n}{test\PYZus{}sj}\PY{p}{)}\PY{o}{.}\PY{n}{astype}\PY{p}{(}\PY{n+nb}{int}\PY{p}{)}
         \PY{n}{test\PYZus{}iq} \PY{o}{=} \PY{n}{test\PYZus{}data}\PY{p}{[}\PY{n}{test\PYZus{}data}\PY{o}{.}\PY{n}{city} \PY{o}{==} \PY{l+s+s1}{\PYZsq{}}\PY{l+s+s1}{iq}\PY{l+s+s1}{\PYZsq{}}\PY{p}{]}\PY{o}{.}\PY{n}{copy}\PY{p}{(}\PY{p}{)}
         \PY{n}{test\PYZus{}iq}\PY{o}{.}\PY{n}{fillna}\PY{p}{(}\PY{n}{method} \PY{o}{=} \PY{l+s+s1}{\PYZsq{}}\PY{l+s+s1}{ffill}\PY{l+s+s1}{\PYZsq{}}\PY{p}{,} \PY{n}{inplace} \PY{o}{=} \PY{k+kc}{True}\PY{p}{)}
         \PY{n}{predictions\PYZus{}iq} \PY{o}{=} \PY{n}{iq\PYZus{}model}\PY{o}{.}\PY{n}{predict}\PY{p}{(}\PY{n}{test\PYZus{}iq}\PY{p}{)}\PY{o}{.}\PY{n}{astype}\PY{p}{(}\PY{n+nb}{int}\PY{p}{)}
         \PY{n}{result} \PY{o}{=} \PY{n}{pd}\PY{o}{.}\PY{n}{read\PYZus{}csv}\PY{p}{(}\PY{l+s+s1}{\PYZsq{}}\PY{l+s+s1}{data/submission\PYZus{}format.csv}\PY{l+s+s1}{\PYZsq{}}\PY{p}{)}
         \PY{n}{result}\PY{o}{.}\PY{n}{total\PYZus{}cases} \PY{o}{=} \PY{n}{np}\PY{o}{.}\PY{n}{concatenate}\PY{p}{(}\PY{p}{[}\PY{n}{predictions\PYZus{}sj}\PY{p}{,} \PY{n}{predictions\PYZus{}iq}\PY{p}{]}\PY{p}{)}
         \PY{n}{result}
         \PY{c+c1}{\PYZsh{}result.to\PYZus{}csv(\PYZdq{}data/predicted\PYZus{}result.csv\PYZdq{})}
\end{Verbatim}



    % Add a bibliography block to the postdoc
    
    
    
    \end{document}
